\documentclass[9pt]{article}
%encoding
%--------------------------------------
\usepackage[utf8]{inputenc}
\usepackage[T1]{fontenc}
%--------------------------------------
 \usepackage{listings} \lstset{numbers=left, numberstyle=\tiny, numbersep=5pt} \lstset{language=Python} 
%Hyphenation rules
%--------------------------------------
\usepackage{hyphenat}
\hyphenation{Mathe-matik wieder-gewinnen}
%--------------------------------------
\usepackage{amsmath, amssymb, mathtools}
\usepackage{mathtools}
%opening
\title{Informatik IV}
\author{}

\newcommand{\specialcell}[2][r]{%
    \begin{tabular}[#1]{@{}c@{}}#2\end{tabular}}

\begin{document}
%\textheight220mm
%\maketitle

%\chapter{Basics}
\section{Grundbegriffe}
\begin{itemize}
	\item Ein Alphabet ist eine endliche, nichtleere Menge $\Sigma$ von Buchstaben  (oder Symbolen).
	\item Ein Wort über $\Sigma$ ist eine endliche Folge von Elementen aus $\sum$.
	\item Die Länge eines Wortes w (bezeichnet mit $|w|$) ist die Anzahl der Symbole in w.
	\item Das leere Wort ist das eindeutig bestimmte Wort der Länge 0 und wird mit dem griechischen Buchstaben $\lambda$ bezeichnet.
	\item Die Menge aller Wörter über $\Sigma$ bezeichnen wir mit $\Sigma^*$.
	\item Eine formale Sprache über $\Sigma$ ist eine jede Teilmenge von $\Sigma^*$
	\item Die leere Sprache ist die Sprache die keine Wörter enthält, und wird mit $\emptyset$ bezeichnet.
	\item die Kardinalität einer Sprache L ist die Anzahl der Wörter von L und wird mit $\|L\|$ bezeichnet.
\end{itemize}
\section{Operationen}
\begin{itemize}
	\item Vereinigung $\{1, 2\} \cup \{2, 3\} = \{1, 2, 3\}$.
	\item Durchschnitt $\{1, 2\} \cap \{2, 3\} = \{2\}$.
	\item Differenz $A - B = \{x \in A\ und\ x \notin B\}$.
	\item Komplement $\overline{A} = \{x \in \Sigma^*|x\notin B\}$.
	\item Konkatenation von Wörtern
	\begin{itemize}
		\item Ist $u = v = \lambda$, so ist $uv = vu = \lambda$.
		\item Ist $v = \Lambda$, so ist $uv = u$.
		\item Ist $u = \lambda$ so ist $uv = v$.
		\item Ist $u = u_1 u_2 \dots u_n und v = v_1 v_2 \dots v_m $ mit $u_i , v_i \in \Sigma$, so ist
		$$uv = u_1 u_2 \dots u_n v_1 v_2 \dots \varepsilon_m.$$
	\end{itemize}
	\item Konkatenation von Sprachen:
	$AB = \{ab|a\in A \ und\ b \in B\}.$
	\item Iteration einer Sprache:
	$A^0 = \{\lambda\},\ A^n = AA^{n-1},\ A^* = \bigcup_{n\geq 0}A^n$.
	\item Spiegelbildoperation von Wort $sp(u)=u_n \dot u_2 u_1$.
	\item Spiegelbildoperation von Sprache $sp(A)=\{sp(w)|w \in A\}$.
	\item Teilwortrelation auf $\Sigma^*$: 
	 $u \sqsupseteq v \leftrightarrow (\exists v_1, v_2 \in \Sigma^*)[v_1 u v_2 = v]$.
	 \item Anfangswortrelation auf $\Sigma^*$:
	 $ u \sqsupseteq_a v \leftrightarrow (\exists w \in \Sigma^*)[uw = v]$.
\end{itemize}
\section{Symbole}
\begin{itemize}
	\item $\Sigma$ ein Alphabet von Terminalsymbolen
	\item $N$ eine Endliche Menge von Nichtterminalen, $\Sigma \cap N = \emptyset$
	\item S Startsymbol, $S \in N$
	\item P Produktionsregeln, $P \subseteq (N \cup \Sigma)^+ \times (N \cup \Sigma)^*$
\end{itemize}
\section{Grammatik}
$G = (\Sigma, N, S, P)$
\begin{itemize}
	\item Typ-0: Ohne Einschränkungen.
	\item Typ-1: $\forall p \rightarrow q \in P : |p| \leq |q|$ 
	\item Typ-2: $\forall p \rightarrow q \in P : p \in N$ 
	\item Typ-3: $\forall p \rightarrow q \in P : |p| \in N \ und\ q \in \Sigma \cup \Sigma N$ 
\end{itemize}
$REG \subseteq CF \subseteq CS \subseteq \mathcal{L}_0$
\subsection{Sonderregelung für $\lambda$}
Typ-i Grammatiken mit $i\in \{1, 2, 3\}$ sind nichtverkürzend, daher $\lambda \notin L(G)$. Daher folgende Sonderregelung:
\begin{enumerate}
	\item Die Regel $S \rightarrow \lambda$ ist als einzige verkürzende Regel für Grammatiken vom Typ 1, 2, 3 zugelassen.
	\item Tritt die Regel $S \rightarrow \lambda$ auf, so darf $S$ auf keiner rechten Seite einer Regel vorkommen.
\end{enumerate}
Dies kann für alle Fälle mit folgender Umwandlung erreicht werden:
\begin{enumerate}
	\item In allen Regeln der Form $S \rightarrow u$ aus $P$ mit $u \in (N \cup \Sigma)^*$ wird jedes Vorkommen von $S$ in $u$ durch ein neues Nichtterminal $S'$ ersetzt.
	\item Zusätzlich enthält $P'$ alle Regeln aus $P$, mit $S$ ersetzt durch $S'$.
	\item Die Regel $S \rightarrow \lambda$ wird hinzugefügt.
\end{enumerate}
\section{Reguläre Sprachen}
\subsection{DFA}
\subsubsection{Definition}
$M=(\Sigma, Z, \delta, z_o, F)$
\begin{itemize}
	\item $\Sigma$: Alphabet
	\item $Z$: endliche Menge von Zuständen mit $\Sigma \cap Z = \emptyset$
	\item $\delta: Z \times \Sigma \rightarrow Z$ Überführungsfunktion
	\item $z_0 \in Z$ Startzustand
	\item $F \subseteq Z$ Endzustände
\end{itemize}
\subsubsection{Beispiel}
	\begin{tabular}{|c||c|c|c|c|}
		\hline
		$\delta$ & $z_o$ & $z_1$ & $z_2$ & $z_3$\\
		\hline \hline
		$0$ & $z_1$ & $z_3$ & $z_2$ & $z_3$\\
		\hline
		$1$ & $z_3$ & $z_2$ & $z_2$ & $z_3$\\
		\hline
	\end{tabular}
\subsubsection{$DFA \rightarrow Grammar$}
	\begin{itemize}
		\item $N = Z$,
		\item $S = z_0$,
		\item P:
		\begin{itemize}
			\item Gilt $\delta(z, a) = z'$, so ist $z \rightarrow az'$ in $P$.
			\item Ist $z' \in F$, so ist zusätzlich $z \rightarrow a$ in $P$.
			\item ist $\lambda \in A$ (d.h., $z_o \in F$), so ist auch $z_0 \rightarrow \lambda$ in $P$, und die bisher konstruierte Grammatik wird gemäß der Sonderregel für $\lambda$ modifiziert.
		\end{itemize}
	\end{itemize}
\subsection{NFA}
$M=(\Sigma, Z, \delta, S, F)$
\begin{itemize}
	\item $\Sigma$: Alphabet
	\item $Z$: endliche Menge von Zuständen mit $\Sigma \cap Z = \emptyset$
	\item $\delta: Z \times \Sigma \rightarrow \mathcal{P}(Z)$: Überführungsfunktionen zur Potenzmenge von Z
	\item $S \subseteq Z$: Menge der Startzustände
	\item $F \subseteq Z$ Menge der Endzustände
\end{itemize}
\subsubsection{$NFA \rightarrow DFA$ (Rabin und Scott)}
NFA $M = (\Sigma, Z, \delta, S, E)$ und DFA $M' = (\Sigma, \mathcal{P}(Z), \delta', z_0', F)$
\begin{itemize}
	\item Zustandsmenge von $M'$: $\mathcal{P}(Z)$,
	\item $\delta'(Z', a)= \cup_{z\in Z'}\delta(z, a)=\hat\delta(Z', a)$ für, all $Z' \subseteq Z$ und $a \in \Sigma$,
	\item $z_o' = S$,
	\item $F = \{Z' \subseteq Z | Z' \cap E \not \emptyset\}$
\end{itemize}
\subsection{$Grammatik \rightarrow NFA$}
\begin{itemize}
	\item $Z = N \cup \{X\}$, wobei $X \notin N \cup \Sigma$ ein neues Symbol ist,
	\item
	$$F =
	 \begin{dcases}
		\{S, X\} &\text{falls }S \rightarrow \lambda\text{ in }P\\
		\{X\}&\text{falls }S \rightarrow \lambda\text{ nicht in }P,
	\end{dcases} $$
	\item $S' = \{S\}$ und
	\item für alle $A \in N$ und $a \in \Sigma$ sei
	$$\delta(A, a) = \left(\bigcup_{A\rightarrow aB \in P} \{B\} \right) \cup \bigcup_{A\rightarrow a \in P} \{X\}.$$
\end{itemize}
\subsection{Regex}
\begin{itemize}
	\item $\emptyset\ und\ \lambda\ sind\ regul\ddot{a}re\ Ausdr\ddot{u}cke$
	\item $Jedes\ a \in \Sigma\ ist\ ein\ regul\ddot{a}rer\ Ausdruck.$
	\item $Sind\ \alpha\ und\ \beta\ regul\ddot{a}re\ Ausdr\ddot{u}cke,\ so\ sind\ auch$
	  	\begin{itemize}
			\item $\alpha\beta$
			\item $(\alpha + \beta)\ und$
			\item $(\alpha)^*$
		\end{itemize}
		$regul\ddot{a}re Ausdr\ddot{u}cke$
	\item $Nichts\ sonst\ ist\ ein\ regul\ddot{a}rer\ Ausdruck$
\end{itemize}
%TODO: REGEX -> NFA, DFA -> REGEX, pages 34-36
\subsection{$NFA \rightarrow L(M)$ Gleichungssysteme}
Bilde ein Gleichungssystem mit $n$ Variablen und $n$ Gleichungen:
\begin{enumerate}
	\item Jedes $z_i \in Z, 1 \leq i \leq n$ ist Variable auf der linken Seite einer Gleichung
	\item Gilt $z_j \in \delta(z_i, a)$ für $z_i, z_j \in Z$ und $a \in \Sigma$, so ist $az_j$ Summand auf der rechten Seite der Gleichung ,,$z_i = \dots$''
	\item Gilt $z_i \in F$, so ist $\emptyset^*$ Summand auf der rechten Seite der Gleichung ,,$z_i = \dots$''.
\end{enumerate}
Todo: Die $z_i$ werden als reguläre Sprachen interpretiert und gemäß Lemma 2.24 und Satz 2.226 ausgerechnet.
Es gilt dann: $L(M) = \bigcup_{z_i \in S}z_i$ bzw. $L(;) = L(\alpha)$ für den regulären Ausdruck $\alpha = \Sigma_{z_i \in S}z_i$.
\subsection{Pumping Lemma REG}
Sei $L \in$ REG. Dann existiert eine (von $L$ abhängige) Zahl $n \geq 1$, so dass sich alle Wörter $x \in L$ mit $|x|\geq n$ zerlegen lassen in $x = uvw$ wobei gilt:
\begin{enumerate}
	\item $|uv| \leq n$,
	\item $|v| \geq 1$,
	\item $(\forall i \geq 0)[uv^iw \in L]$.
\end{enumerate}
\subsection{Mihill Nerode Minimalautomaten}
\subsubsection{Mihill Nerode Relation}
$xR_Ly$ zwichen $x$ und $y$ gild genau dann, wenn $(\forall z \in \Sigma^*)[xz\in L \leftrightarrow yz \in L]$.
Dies induziert eine Zerlegung von $\Sigma^*$ in Äquivalenzklassen: $$[x] = \{y \in \Sigma^* | xR_Ly\}$$
Die Anzahl der Äquivalenzklassen ist $\text{Index}(R_L) = \|\{[x]|x \in \Sigma^*\}\|$.
$$L\in REG \leftrightarrow \text{Index}(R_L) < \infty$$
\textbf{\underline{Algorithmus}}\\
\textbf{Eingabe:} DFA $M = (\Sigma, Z, \delta, z_0, F)$.\\
\textbf{Ausgabe:} Ein zu $M$ äquivalenter Minimalautomat\\
\textbf{Schritte}:
\begin{enumerate}
	\item Entferne alle von $z_0$ aus nicht erreichbaren Zustände aus $Z$.
	\item Erstelle eine Tabelle aller (ungeordneten) Zustandspaare $\{z, z'\}$ on $M$ mit $z \neq z'$.
	\item Markiere alle Paare $\{z, z'\}$ mit
	$z\in F \leftrightarrow z' \notin F$.
	\item Seit $\{z, z'\}$ ein unmarkiertes paar. Prüfe für jedes $a \in \Sigma$, ob
	$\{\delta(z, a), \delta(z', a)\}$
	bereits markiert ist. Ist mindestens ein Test erfolgreich, so markiere auch $\{z, z'\}$.
	\item Wiederhole Schritt 4, bis keine Änderung mehr eintritt.
	\item Bilde maximale Mengen paarweise nicht disjunkter unmarkierter Zustandspaare und verschmelze jeweils alle Zustände einer Menge zu einem neuen Zustand.
\end{enumerate}
\subsection{Abschlusseigenschaften Definitionen}
\begin{enumerate}
	\item Vereinigung, falls $(\forall A, B \subseteq \Sigma^*)[(A \in \mathcal{C} \wedge B \in \mathcal{ C}) \implies A \cup B \in \mathcal{C}]$;
	\item Komplement, falls $(\forall A \subseteq \Sigma^*)[A \in \mathcal{C} \implies \overline{A}\in\mathcal{C}]$;
	\item Schnitt, falls $(\forall A, B \subseteq \Sigma^*)[(A \in \mathcal{C} \wedge B \in \mathcal{C}) \implies A \cap B \in \mathcal{C}]$;
	\item Differenz, falls $(\forall A, B \subseteq \Sigma^*)[(A \in \mathcal{C} \wedge B \in C) \implies A \cap B \in \mathcal{C}]$;
	\item Konkatenation, falls $(\forall A, B \subseteq \Sigma^*)[(A \in \mathcal{C} \wedge B \in \mathcal{C})\implies AB \in \mathcal{C}]$;
	\item Iteration (Kleene-Hülle), falls $(\forall A \subseteq \Sigma^*)[A\in\mathcal{C}\implies A^*\in\mathcal{C}]$;
	\item Spiegelung, falls $(\forall A\subseteq \Sigma^*)[A\in\mathcal{C}\implies sp(A)\in\mathcal{C}]$;
\end{enumerate}
\subsection{Characterisierung}
% TODO: REMOVE subsection
\begin{enumerate}
	\item Es gibt eine rechtslineare Grammatik $G$ mit $L(G) = L$.
	\item Es gibt eine linklineare Grammatik $G$ mit $L(G)=L$.
	\item Es gibt einen DFA $M$ mit $L(M)=L$.
	\item Es gibt einen NFA $M$ mit $L(M)=L$.
	\item Es gibt einen regulären Ausdruck $\alpha$ mit $L(\alpha)=L$.
	\item Für die Myhill-Nerode-Relation $R_L$ gilt: $\text{Index}(R_L)<\infty$.
\end{enumerate}
\section{Kontextfreie Sprachen}
\subsection{Normalformen}
Ausnahmeregel für das leere Wort muss nur für Typ-2 Grammatiken nicht genutzt werden. Eine $kfg G = (\Sigma, N, S, P)$ heißt $\lambda$-frei, falls in $P$ keine Regel $A \to \lambda$ mit $A \neq S$ auftritt. Diese Umwandlung ist immer möglich.
\subsubsection{kfG $\to$ $\lambda$-frei}
Wenn $\lambda \in L(G)$ wende Sonderregelung für $\lambda$ an.
\begin{enumerate}
	\item Bestimme die Menge $N_\lambda=\{A \in N | A \vdash^*_G \lambda\}$ sukzessive wie folgt:
	\begin{enumerate}
		\item Ist $A \to \lambda$ eine Regel in $P$, so ist $A \in N_\lambda$.
		\item Ist $ A \to A_1 A_2 \dots A_k $ eine Regel in $P$ mit $k \geq 1$ und $A_i \in N_\lambda$ für alle $i$, $1 \leq i \leq k$, so ist $A \in N_\lambda$.
	\end{enumerate}
	\item Füge für jede Regel der Form
	$$B \to uAv \text{ mit } B \in N, A\in N_\lambda \text{ und } uv \in (N \cup \Sigma)^+$$
	zusätzlich die Regel $B \to uv$ zu $P$ hinzu.
	\item entferne alle Regeln $A \to \lambda$ aus $P$.
\end{enumerate}
Schritt 2 muss auch für neu generierte Regeln iterativ angewendet werden. Dies ergibt die gesuchte $\lambda$-freie kfG $G'$ mit $L(G) =L(G')$.
\subsubsection{Einfache Regeln entfernen}
Regeln $A \to B$ heißen einfach falls $A, B \in N$
\begin{enumerate}
	\item Entferne alle Zyklen
	$$B_1 \to B_2, B_2\to B_3, \dots , B_{k-1} \to B_k, B_k \to B_1 \text{ mit } B_i \in N$$
	und ersetze all $B_i$ (in den verbleibenden Regelnd) durch ein neues Nichtterminal B.
	\item Nummeriere die Nichtterminale als $\{A_1, A_2, \dots A_n\}$ so, dass aus $A_i \to A_j$ folgt: $i < j$.
	\item Für $k = n - 1, n - 2, \dots , 1$ (RÜCKWÄRTS!) eliminiere die Regel $A_k \to A_l$ mit $k < l$ so: Sind die Regeln mit $A_l$ als linker Seite gegeben durch
	$$A_l \to u_1 | u_2 | \dots | u_m,$$
	so entferne $A_k \to A_l$ und füge die folgenden Regeln hinzu:
	$$A_k \to u_1 | u_2 | \dots | u_m.$$
\end{enumerate}
\subsubsection{$\lambda$-frei ohne einfache Regeln $\to$ Chomsky-Normalform}
\textbf{Bedingung}: Alle Regeln in P haben die form
\begin{itemize}
	\item $A \to BC$ mit $A, B, C \in N$;
	\item $A \to a$ mit $A \in N$ und $a \in \Sigma$.
\end{itemize}
\textbf{Überführung:}
\begin{enumerate}
	\item Regeln $A \to a$ mit $A \in N$ und $a \in \Sigma$ sind $CNF$ und werden übernommen. Alle anderen Regeln sind von der Form: $A \to x$ mit $x \in (N \cup \Sigma)^*$ und $|x| \geq 2$.
	\item Füge für jedes $a \in \Sigma$ ein neues Nichtterminal $B_a$ zu $N$ hinzu, ersetze jedes Vorkommen von $a \in \Sigma$ durch $B_a$ und füge zu $P$ die Regel $B_a \to a$ hinzu.
	\item Nicht in $CNF$ Sind nun nur noch Regeln der Form
	$$A \to B_1B_2\dots B_k\text{, wobei } k \geq 3 \text{ und jedes } B_i \text{ein Nichtterminal ist.}$$
	Jede solche Regel wird ersetzt durch die Regeln:
	
	\begin{align*}
	A &\to B_1 C_2,\\
	C_2 &\to B_2 C_3,\\
    &\vdotswithin{\to}\\
	C_{k-2} &\to B_{k-2}C_{k-1},\\
	C_{k-1} &\to B_{k-1}B_k,
	\end{align*}
	wobei $C_2, C_3, \dots , C_{k-1}$ neue Nichtterminale sind.
\end{enumerate}
Dies liefert die gesuchte Grammatik $G'$ in CNF mit $L(G) = L(G')$
\subsubsection{Bemerkung CNF}
\begin{itemize}
	\item Ableitung von $w$ in $G$ in genau $2|w|-1$ Schritten.
	\item Syntaxbaum ist ein Binärbaum.
\end{itemize}
\subsubsection{Greibach-Normalform}
Zu jeder kfG mit $\lambda \notin L(G) gibt es eine kfG G' in GNF, so dass L(G) = L(G')$ und jede Regel in Folgender Form ist:
$$A \to aB_1B_2 \dots B_k \text{ mit } k \geq 0 \text{ und } a \in \Sigma$$
\subsubsection{Bemerkung GNF}
Man unterscheidet bezüglich der Länge der rechten Seite der Produktion:
\begin{itemize}
	\item Jede kfG kann in eine äquivalente kfG in Greibach-Normalform transformiert werden, so dass für alle Regeln $A \to aB_1B_2 \dots B_k$ stets $k \leq 2$ gilt.
	\item Für den Spezialfall $k \in \{0, 1\}$ erhalten wir gerade die Definition rechtslinearer Grammatiken.
	\item Die Ableitung eines Wortes $w \in L(G), w \neq \lambda$ ist genau $|w|$ Schritte lang wenn G in GNF steht.
\end{itemize}
\subsection{Pumping-Lemma CF}
Sei $L$ eine kontextfreie Sprache. Dann existiert eine (von L abhängige) Zahl $n \ge 1$, so dass sich alle Wörter $z \in L$ mit $|z| \ge n$ zerlegen lassen in $z = uvwxy$, wobei gilt:
\begin{enumerate}
	\item $|vx| \ge 1$.
	\item $|vwx \le n$.
	\item $(\forall i \ge 0)[uv^iwx^iy \in L]$.
\end{enumerate}
\subsection{Satz von Parikh}
\subsubsection{Parikh Abbildung}
$\Psi: LL \to \mathcal{N}^n$ ist definiert durch
$$\Psi(w) = (|w|_a1, |w|_a2, \dots, |w|_an),$$
wobei $|w|_a$ die Anzahl der Vorkommen des Zeichens $a$ in $w$ angibt. Für Sprachen:
$\Psi: \mathcal{P}(\Sigma^*) \to \mathcal{P}(\mathcal{N}^n)$ ist definiert durch
$$\Psi(L) = \{\Psi(w)|w \in L\}$$
Eine Menge ist semilinear wenn sie aus linearen Mengen zusammengesetzt ist.
\subsubsection{Parikh}
Für jede kontextfreie Sprache $L$ in $\Psi(L)$ ist semilinear
\subsubsection{Beispiel}
$$L = \{x \in \{0, 1\}^i | x = 1^k \text{ mit } k \ge 0 \text{ oder }  x = 0^j1^{k^2} mit j \ge 1 \text{ und } k \ge 1\}$$
über $\Sigma = \{0, ,1\}$. Die Menge
$$ \Psi(L) = \{(i, j)|(i = 0\text{ und }j \ge 0) \text{ oder } (i \ge 1 \text{ und } j = k^2 \text{ und } k \ge 1)\}$$
ist nicht semilinear und L somit nicht kontextfrei.
\subsection{Abschlusseigenschaften CF}
CF ist abgeschlossen unter Vereinigung, Konkatenation, Iteration, Spiegelung. CF ist abgeschlossen unter Schnitt mit REG.
\subsection{CYK Algorithmus}
Überprüfe ob Wort in Sprache. Zeile i, Spalte j der Tabelle ist
 \begin{lstlisting}
for k in 0..<j:
  t[i, j].add nonterminals_with_rule_to_pair( (i, k), (i+k+1, -k))
 \end{lstlisting}
 Wenn S in der letzten Zeile steht is $w \in L(G)$.
 \subsection{Kellerautomaten (PDA)}
 \subsubsection{Definition}
 $M = (\Sigma, \Gamma, Z, \delta, z_0, \#)$
 \begin{itemize}
 	\item $\Sigma$: Eingabe-Alphabet
 	\item $\Gamma$: Kelleralphabet
 	\item Z: endliche Menge von Zuständen
 	\item $\delta: Z \times (\Sigma \cup \{\lambda\}) \times \Gamma \to \mathcal{P}_e(Z \times \Gamma^+)$ die Überführungsfunktion, $\mathcal{P}_e(Z \times \Gamma^+)$ ist die Menge aller endlichen Teilmengen von $Z \times \Gamma^+$
 	\item $z_0 \in Z$: Startzustand
 	\item $\# \in \Gamma$ das Bottom-Symbol im Keller
 \end{itemize}
 $\delta$-Übergänge $(z', B_1,B_2 \dots B_k)\in \delta(z, a, A)$ schreiben wir kurz auch $zaA \to z'B_1B_2\dots B_k$
 \begin{itemize}
 	\item Akzeptiert wenn Keller leer
 	\item Schritte ohne Lesen eines Eingabesymbols  sind in der Form $z\lambda A \to z'B_1B_2 \dots B_k$ möglich.
 	\item Es ist nur ein Startzustand nötig
 \end{itemize}
 Sprache:
 \begin{itemize}
 	\item  $\mathcal{k}_m = Z \times \Sigma^* \times \Gamma^*$ ist die Menge aller Konfigurationen von M
 	\item ist $k = (z, \alpha, \gamma)$ eine Konfiguration aus $\mathcal{k}_M$, so ist im aktuallen Takt der Rechnung von M:
 	\begin{itemize}
 		\item $z \in Z$ der aktuelle Zustand von M;
 		\item $alpha \in \Sigma^*$ der noch zu lesende Teil des Eingabeworts;
 		\item $\gamma \in \Gamma^*$ der aktuelle Kellerinhalt.
 	\end{itemize}
 	Für jedes Eingabewort $w \in \Sigma^*$ ist $(z_0, w, \#)$ die entsprechende Startkonfiguration von M.
 	\item Auf $\mathcal{k}_m$ definieren wir eine binäre Relation $\vdash_m \subseteq \mathcal{k}_M \times \mathcal{k}_M$ wie folgt:
 	\begin{itemize}
 		\item Für $k, k' \in \mathcal{k}_M$ gilt $k \vdash_m k'$ genau dann, wenn $k'$ aus $k$ durch eine Anwendung von $\delta$ hervorgeht.
 	\end{itemize}
 	\item $\vdash^*_M$ ist die reflexive und transitive Hülle von $\vdash_M$
 	\item Die vom PDA $M$ akzeptierte Sprache ist definiert durch
 	$$L(M) = \{w \in \Sigma^*|(z_0, w, \#)\vdash^*_M (z, \lambda, \lambda) \text{ für ein } z \in Z\}$$
 	$(z, \lambda, \lambda)$ ist dann eine Endkonfiguration für PDA M.
 \end{itemize}
 \subsubsection{Kontextfrei $\to$ PDA}
 Von $G = (\Sigma, N, S, P)$ zu $M = (\Sigma, N\cup\Sigma, \{z\}, \delta, z, S)$:
 \begin{enumerate}
 	\item Ist $A \to q$ eine Regel in $P$ mit $A \in N$ und $q \in (N \cup \Sigma)^+$, so sei $(z, q) \in \delta(z, \lambda, A)$.
 	\item Für jedes $a \in \Sigma$ sie $(z, \lambda), \in \delta(z, a, a)$.
 \end{enumerate}
\subsubsection{PDA $\to$ Kontextfrei}
 Von $M = (\Sigma, \Gamma, Z, \delta, z_0, \#)$ zu $G = (\Sigma, \{S\} \cup Z \times \Gamma \times Z, S, P)$:\\
 O.B.d.A Für alle $\delta$-Regeln der Form $zaA \to z'B_1B_2\dots B_k$ gelte $k \le 2$.
 P besteht dann aus den folgenden Regeln:
 \begin{enumerate}
 	\item $S \to (z_0, \#, z)$ für jedes $z \in Z$.
 	\item $(z, A, z') \to a$, falls $(z', \lambda) \ in \delta(z, a, A)$.
 	\item $(z, A, z') \to a(z_1, B, z')$ falls $(z_1), B) \in \delta(z, a, A)$.
 	\item $z, A, z') \to a(z_1, B, z_2)(z_2, C, z')$, falls $z_1, BC) \in \delta(z, a, A)$.
 \end{enumerate}
 Wobei $z, z', z_1, z_2 \in Z, A, B, C \in \Gamma$ und $a \in \Sigma \cup \{\lambda\}$.
\section{Deterministiche Kontextfreie Sprache, DCF}
$M = (\Sigma, \Gamma, Z, \delta, z_0, \#, F)$
\begin{enumerate}
	\item $M' = (\Sigma, \Gamma, Z, \delta, z_0, \#)$ ist PDA
	\item $(\forall a \in \Sigma)(\forall A \in \Gamma)(\forall z \in Z)[\|\delta(z, a, A)\| + \|\delta(z, \lambda, A)\| \le 1]$;
	\item $F \subseteq Z$ ist eine ausgezeichnete Teilmenge von Endzuständen (M akzeptiert per Endzustand, nicht per leerem Keller)
\end{enumerate}
Die akzeptierte Sprache ist
$$L(M) = \{x \in \Sigma^*|(z_0, x, \#)\vdash^*_M (z, \lambda, \gamma) \text{ für ein } z \in F \text{ und } \gamma \in \Gamma^*\}$$
Eine Sprache heißt Deterministisch kontextfrei wenn es einen deterministischen Kellerautomaten M gibit mit $A = L(M)$.
\subsubsection{Abschlusseigenschaften DCF}
DCF ist abgeschlossen unter Komplement
% LR und LL GRAMMATIKEN TODO
\section{$\mathcal{L}_0$ Sprachen (Turingmaschinen)}
$M = (\Sigma, \Gamma, Z, \delta, z_0, \square, F)$\\
Mit k Bändern:
\begin{itemize}
	\item $\Sigma$: Eingabe-Alphabet,
	\item $\Gamma$: Arbeitsalphabet mit $\Sigma \subseteq \Gamma$,
	\item $Z$: endliche Menge von Zuständen mit $Z \cap \Gamma = \emptyset$,
	\item $\delta: Z \times \Gamma^k \to \mathcal{P}(Z \times \Gamma^K \times \{L, R, N\}^k)$: Überführungsfunktion,
	\item $z_0 \in Z$: Startzustand,
	\item $\square \in \Gamma - \Sigma$: ,,Blank''-Symbol
	\item $F \subseteq Z$: Menge der Endzustände
\end{itemize}
Spezialfall der deterministischen Turingmaschine mit k Bändern mit k Bändern ergibt sich, wenn $\delta$ von $Z \times \Gamma^k$ nach $Z \times \Gamma^k \times \{L, R, N\}^k$ abbildet. Für $k = 1$ ergibt sich die 1-Band-Turingmaschine die mit TM abgekürzt wird. Jede Turingmaschine kann durch eine TM simuliert werden.
Statt $(z', b, x)\in \delta(z, a)$ mit $z, z' \in Z, x \in \{L, R, N\}, a, b \in \Gamma$ schreiben wir kurz: $(z, a) \to (z', b, x)$, oder $za \to z'bx$
% Semantik von Turingmaschinen TODO
\subsection{Linear beschränkte Automaten (LBA)}
\begin{enumerate}
	\item Verdoppele das Eingabe-Alphabet $\Sigma$ mit $\hat{\Sigma} = \Sigma \cup \{\hat{a}|a \in \Sigma\}$
	\item Repräsentiere die Eingabe $a_1a_2 \dots a_n \in \Sigma^+$ durch das Wort $a_1a_2 \dots a_{n-1}\hat{a_n}$ über $\hat{\Sigma}$
		
\end{enumerate}
\begin{itemize}
	\item Eine nichtdeterministische TM $M$ heißt linear beschränkter Automat (kurz LBA), fals für all Konfigurationen $\alpha z \beta$ und
	\begin{itemize}
		\item für alle Wörter $x = a_1a_2 \dots a_{n-1}a_-n \in \Sigma^+$ mit
		$$z_0a_1a_2 \dots a_{n-1}\hat{a_n}\vdash^*_m \alpha z \beta$$
		gilt: $|\alpha \beta| = n$, und
		\item für $x = \lambda$ mit $z_0 \square \vdash^*_M \alpha z \beta$ gilt: $\alpha\beta = \square$
	\end{itemize}
	\item Die vom LBA $M$ akzeptierte Sprache ist definiert durch
	$$L(M) = \left\{
					a_1a_2 \dots a_{n-1}a_n \in \Sigma^* \middle|
					\begin{split}
					&z_0a_1a_2 \dots a_{n-1} \hat{a_n} \vdash^*_m \alpha z\beta\\
					&\text{mit } z \in F \text{ und } \alpha, \beta \in \Gamma^*
					\end{split}
		   	\right\}$$
\end{itemize}
\subsubsection{CS $\to$ LBA}
Von $G = (\Sigma, N, S, P)$ zu $M = (\Sigma, \Gamma, Z, \delta, z_0, \square, F)$
\begin{enumerate}
	\item Eingabe $x = a_1a_2 \dots a_n$.
	\item Wähle nichtdeterministisch eine Regel $u \to v$ aus $P$ und suche eine beliebiges Vorkommen von $v$ in der aktuellen Bandinschrift von $M$.
	\item Ersetzt $v$ durch $u$. Ist dabei $|u| < |v|$, so verschiebe entsprechend alle Symbole rechts der Lücke um diese zu schließen.
	\item Ist die aktuelle Bandinschrift nur noch das Startsymbol $S$, so halte im Endzustand und akzeptiere; andernfalls gehe zu (2) und wiederhole.
\end{enumerate}
\subsubsection{LBA $\to$ CS}

Von $M = (\Sigma, \Gamma, Z, \delta, z_0, \square, F)$ und $G= (\Sigma, \{S, A\} \cup \{\Delta \times\ \Sigma\}, S, P)$ mit
$$\Delta = \Gamma \cup (Z \times \Gamma)$$
$$\vdots$$$$\vdots$$
%TODO
\subsubsection{$\mathcal{L_0} \to$ TM}
\begin{enumerate}
	\item Eingabe $x = a_1a_2 \dots a_n$.
	\item Wähle nichtdeterministisch eine Regel $u \to v$ aus $P$ und suche eine beliebiges Vorkommen von $v$ in der aktuellen Bandinschrift von $M$.
	\item Ersetzt $v$ durch $u$. Ist dabei $|u| < |v|$, so verschiebe entsprechend alle Symbole rechts der Lücke um diese zu schließen.
	\item Ist die aktuelle Bandinschrift nur noch das Startsymbol $S$, so halte im Endzustand und akzeptiere; andernfalls gehe zu (2) und wiederhole.
\end{enumerate}
\begin{table}[h]
	\begin{tabular}{|c||c|}
		\hline
	     Typ 3 & \specialcell[t]{reguläre Grammatik\\ deterministhischer endlicher Automat(DFA)\\ nichtdeterministischer endlicher Automat (NFA) \\ regulärer Ausdruck (REG)}\\
		\hline
	\end{tabular}
\end{table}

\end{document}